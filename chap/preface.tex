\chapter{緒言}
\label{ch_preface}
\markboth{\ref{ch_preface}.緒言}{\ref{ch_preface}.緒言}

\section{本研究の背景}
\label{preface_background}
Burgersベクトル\index{ばーがーすべくとる@Burgersベクトル}は,現配置において式\eqref{eq_bergers_vector}で表される~\cite{kondo1955non-riemannian}.
%
\begin{equation}
  \label{eq_bergers_vector}
    \tilde{B}{}^{a}
    = -\int_{\mathscr{S}} \left[ \hat{T}{}^{..a}_{bc} + \frac{1}{2} C^{\ell{}-1ae}
      \left( \hat{R}{}_{bc[de]} + \hat{R}{}_{bc(de)} \right) x^{d} \right] \mathrm{d} x^{b} \wedge \mathrm{d} x^{c}
  \dotfill
\end{equation}
%
これを図示すると,図\ref{fig_paper_irasutoya}のようになる.
%
\begin{figure}[bp]
    \centering
    \includegraphics{./figure/document_ronbun_taba.png}
    \caption{Paper.}
    \label{fig_paper_irasutoya}
\end{figure}
%
